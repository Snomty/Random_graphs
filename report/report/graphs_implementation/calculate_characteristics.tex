\input code-format.tex

В данном разделе представлены ключевые функции для анализа свойств графов. Все функции принимают на вход объект класса \texttt{Graph} и возвращают числовые характеристики.

\begin{lstlisting}
def calculate_min_deg(G: Graph) -> int:
    """ Returns the minimum degree of a graph vertex """

def calculate_max_deg(G: Graph) -> int:
    """ Returns the maximum degree of a graph vertex """

def calculate_number_component(G: Graph) -> int:
    """ Returns the number of connected components of a graph """

def calculate_number_articul(G: Graph) -> int:
    """ Returns the number of articulation points of a graph """

def calculate_number_triangle(G: Graph) -> int:
    """ Returns the number of triangles in a graph """

def calculate_clique_number(G: Graph) -> int:
    """ Returns the click count of a graph """

def calculate_maxsize_independed_set(G: Graph) -> int:
    """ Returns the size of the maximum independent set """
\end{lstlisting}

\noindent Все функции реализованы с помощью библиотеки \textbf{NetworkX}. Почти все функции честным перебором дают точные значения характеристик.\\
Исключениями являются:

\begin{itemize}
    \item Функция подсчета кликового числа. Реализована через жадный поиск хроматического числа графа, основано на том, что для дистанционного графа они совпадают почти наверное.

    \item Функция подсчета числа независимости графа. Реализована через подсчет кликового числа для дополнения.
\end{itemize}

\noindentКаждая функция покрыта тестами.
