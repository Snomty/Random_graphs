
\noindent \textbf{Итоги Ивановой А.А.}

\begin{table}[h]
\centering
\begin{tabular}{lcc}
\toprule
\textbf{Размер выборки} & \textbf{Ошибка I рода ($\alpha$)} & \textbf{Мощность критерия} \\
\midrule
N = 25   & 0.21 & 0.7086 \\
N = 100  & 0.06 & 0.9153 \\
N = 500  & 0.00 & 1.0000 \\
\bottomrule
\end{tabular}
\label{tab:power_analysis}
\end{table}

\noindent \textbf{Общий анализ:}\\
На 500 вершинах можно точно отличить распределения, потому что у них сильно отличаются характеристики, но это достаточно долго, поэтому лучше выбрать 100 вершин, ошибка первого рода 0.06 и 0.03 соответственно, мощность 0.92 и 0.97 соответственно, но при этом обсчет 10к графов займет всего 10 минут\\

\noindent \textbf{Общий вывод:}\\
Классификатор имеет высокую мощность и маленькую ошибку первого рода на 3000 вычислениях, что делает его отличным статистическим критерием.
\noindent